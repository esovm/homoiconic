%-----------------------------------------------------------------------------
%
%               Template for sigplanconf LaTeX Class
%
% Name:         sigplanconf-template.tex
%
% Purpose:      A template for sigplanconf.cls, which is a LaTeX 2e class
%               file for SIGPLAN conference proceedings.
%
% Guide:        Refer to "Author's Guide to the ACM SIGPLAN Class,"
%               sigplanconf-guide.pdf
%
% Author:       Paul C. Anagnostopoulos
%               Windfall Software
%               978 371-2316
%               paul@windfall.com
%
% Created:      15 February 2005
%
%-----------------------------------------------------------------------------


\documentclass[preprint]{sigplanconf}

% The following \documentclass options may be useful:

% preprint      Remove this option only once the paper is in final form.
% 10pt          To set in 10-point type instead of 9-point.
% 11pt          To set in 11-point type instead of 9-point.
% numbers       To obtain numeric citation style instead of author/year.

%%%%%%%%%%%%%%%%%%%%%%%%%%%%%%%%%%%%%%%%

\usepackage{amsmath}

%%%%%%%%%%%%%%%%%%%%%%%%%%%%%%%%%%%%%%%%

\usepackage{amsthm}
\makeatletter
\def\th@definition{%
  \thm@notefont{}% same as heading font
  \normalfont % body font
}
\makeatother
\theoremstyle{definition}
\newtheorem{problem}{Problem}

%%%%%%%%%%%%%%%%%%%%%%%%%%%%%%%%%%%%%%%%

\usepackage{courier}
\usepackage{listings}
\lstset{
    language=Haskell,
    basicstyle=\ttfamily,
    keywordstyle=\ttfamily,
    literate={->}{{$\rightarrow$}}2 {>=}{{$\geq$}}2 {<-}{{$\leftarrow$}}2
             {<=}{{$\leq$}}2 {=>}{{$\Rightarrow$}}2
    %keywordstyle=\lst@ifdisplaystyle\color{blue}\fi,
    %commentstyle=\color{gray}
}
\lstMakeShortInline|

\newcommand{\cL}{{\cal L}}

%%%%%%%%%%%%%%%%%%%%%%%%%%%%%%%%%%%%%%%%%%%%%%%%%%%%%%%%%%%%%%%%%%%%%%%%%%%%%%%%

\begin{filecontents}{paper.bib}
@article{goldberg1991,
  title={What every computer scientist should know about floating-point arithmetic},
  author={Goldberg, David},
  journal={ACM Computing Surveys (CSUR)},
  volume={23},
  number={1},
  pages={5--48},
  year={1991},
  publisher={ACM}
}

@book{higham2002,
  title={Accuracy and stability of numerical algorithms},
  author={Higham, Nicholas J},
  year={2002},
  publisher={Siam}
}

@article{swierstra2008,
  title={Data types {\`a} la carte},
  author={Swierstra, Wouter},
  journal={Journal of functional programming},
  volume={18},
  number={04},
  pages={423--436},
  year={2008},
  publisher={Cambridge Univ Press}
}

@book{hamming2012,
  title={Numerical methods for scientists and engineers},
  author={Hamming, Richard},
  year={2012},
  publisher={Courier Corporation}
}

@inproceedings{gupta2015,
  title={Deep Learning with Limited Numerical Precision},
  author={Gupta, Suyog and Agrawal, Ankur and Gopalakrishnan, Kailash and Narayanan, Pritish},
  booktitle={Proceedings of the 32nd International Conference on Machine Learning (ICML-15)},
  pages={1737--1746},
  year={2015}
}
\end{filecontents}
\immediate\write18{bibtex paper}

%%%%%%%%%%%%%%%%%%%%%%%%%%%%%%%%%%%%%%%%%%%%%%%%%%%%%%%%%%%%%%%%%%%%%%%%%%%%%%%%

\begin{document}

\special{papersize=8.5in,11in}
\setlength{\pdfpageheight}{\paperheight}
\setlength{\pdfpagewidth}{\paperwidth}

\conferenceinfo{CONF 'yy}{Month d--d, 20yy, City, ST, Country}
\copyrightyear{20yy}
\copyrightdata{978-1-nnnn-nnnn-n/yy/mm}
\copyrightdoi{nnnnnnn.nnnnnnn}

% Uncomment the publication rights you want to use.
%\publicationrights{transferred}
%\publicationrights{licensed}     % this is the default
%\publicationrights{author-pays}

\titlebanner{Preprint}        % These are ignored unless
\preprintfooter{Preprint}   % 'preprint' option specified.

%\title{What's the point of homoiconicity?}
\title{The (floating) point of homoiconicity}
\subtitle{Functional Pearl}

\authorinfo{Mike Izbicki}
           {UC Riverside}
           {mike@izbicki.me}
%\authorinfo{Name2\and Name3}
           %{Affiliation2/3}
           %{Email2/3}

\maketitle

\begin{abstract}
%This pearl uses Haskell's type classes to enable a limited form of \emph{homoiconicity}.
%%Specifically, we show how to convert any Haskell98 function that is parametric in its return type into an isomorphic abstract syntax tree.
%Specifically, we show how to convert a certain set of polymorphic functions into isomorphic abstract syntax trees.
%But this is really a pearl about the story of numerical programming in Haskell.

%Numeric programming in Haskell sucks.
%Actually, it sucks in every language because it requires floating point numbers,
%and floating point arithmetic is so difficult to work with.
%The goal of this pearl is to make numeric programming suck just a little bit less.
%This statement is tautologically true---numeric programming requires floating point arithmetic, and working with floats sucks in every language.
%I think this makes numeric programming in Haskell suck just a little bit less.

Floating point numbers are dangerous.
Due to their limited precision,
seemingly correct mathematical formulae can give wildly inaccurate results.
This pearl uses Haskell's type system to mitigate this danger.

The main observation is that many numerical Haskell functions are \emph{homoiconic}.
That is, we can easily convert them into isomorphic abstract syntax trees (ASTs).
The construction does not use template Haskell, and so works at both compile and run time.
We use these ASTs to solve two problems with floating point numbers:
\begin{enumerate}
\item
We show how to write functions that automatically stabilize (and optimize!) mathematical formulae.
This lets the programmer write naive floating point code;
the compiler will worry about improving the numeric stability and speed later.
\item
We show how to specify type class laws for floating point numbers that can be easily and automatically tested.
This is a difficult unsolved problem because floating point numbers do not obey many of the traditional algebraic laws like associativity.
\end{enumerate}
We demonstrate these techniques using GHC's builtin numeric hierarchy,
but these techniques generalize to user defined numeric hierarchies as well.

%First, we show how to convert a certain type of commonly used mathematical function into isomorphic abstract syntax trees.
%Specifically, we show that a subset of Haskell functions (which we call \emph{homogeneous}) can easily be converted into an isomorphic abstract syntax tree.

%I hate numerical programming in Haskell,
%but I hate numerical programming in every language.
%It requires floating point numbers,
%which
%
%I hate floating point numbers.

%This pearl provides a technique that makes working with approximate algebraic structures more convenient.
%The prototypical example of an approximate structure is floating point arithmetic, which is famously not a semigroup because it is not associative.
%The technique has two parts.
%First, it lets us construct an FAlgebra (and hence a syntax tree) for a numerical class hierarchy.
%Second, we show how to treat topology as an algebraic structure, and that floating point types have a useful topology.
%This topology can be used to automatically construct properties for approximate algebraic structures that can be easily checked using quickcheck or smallcheck.
\end{abstract}

\category{CR-number}{subcategory}{third-level}

% general terms are not compulsory anymore,
% you may leave them out
%\terms
%term1, term2

\keywords
floating point, homoiconic, type classes

\section{Introduction}

Imagine you're writing a machine learning library in Haskell.
At some point, you'll need to implement the log-logistic function
\begin{equation}
f(x) = \log\left(\frac{1}{1+\exp{(-x)}}\right)
\label{eq:log-logistic}
\end{equation}
This function is used in logistic regression and many families of neural networks.
As a first attempt, you might write:
\begin{lstlisting}
logLogistic1 :: Floating x => x -> x
logLogistic1 x = log(1/(1+exp(-x)))
\end{lstlisting}
Unfortunately, this straightforward implementation is numerically unstable.
When specialized to the |Double| type,
the |exp| function overflows to infinity for inputs greater than (approximately) $709$.
%This overflow causes |logLogistic| to take the |log| of 0, which is undefined.
Therefore, on an input of $-710$,
|logLogisitic1| will return negative infinity when the true answer is actually very close to $-710$.
Such an inaccurate answer is unacceptable,
so our function must be rewritten.

Writing numerically stable functions is a notoriously difficult black art.
(See \cite{goldberg1991,higham2002,hamming2012} for some classic introductions.)
Fortunately, a simple technique can be used to improve the stability of the log-logistic function.
First, check our input to see if it would overflow the |exp| function.
If no overflow would occur, then calculate our function as normal;
but if overflow would occur, then use an approximation that does not rely on a call to |exp|.
Performing this transformation gives us:
\begin{lstlisting}
logLogistic2 :: (Floating x, Ord x) => x->x
logLogistic2 x = if x < (-709)
    then x
    else log(1/1+exp(-x))
\end{lstlisting}
%This new function is numerically stable, but we've introduced two new problems:
We've made our function numerically stable, but we've introduced two new problems:

\begin{problem}[obfuscation]
The purpose of the if statement in |logLogistic2| is non-obvious.
To understand its purpose,
the reader must carefully consider the limits of floating point arithmetic.
Anyone who doesn't regularly work with floating point numbers is unlikely to discover this purpose on their own.
By transforming our function to have the correct floating point semantics,
we've obfuscated the function's purpose and made it harder for others to understand and contribute to our library.
\end{problem}

\begin{problem}[specialization]
The \lstinline{exp} function when specialized to \lstinline{Float}s overflows on values greater than (approximately) 88.
This means that |logLogistic2| will give incorrect answers for inputs between 88 and 709.
To properly handle both |Float|s and |Double|s,
we must write two separate functions.
In practice, we may want to support many other numeric types as well.
Half precision (16 bit) floats and fixed point arithmetic are becoming increasingly popular in machine learning applications due to their increased speed, smaller memory footprint, and lower energy consumption \cite{gupta2015}.
We need to manually reimplement the log-logistic function specialized to every numeric type we want our library to support.
\end{problem}

These problems are not unique to the log-logistic function.
All functions over the real or complex numbers need to be transformed for floating point stability,
and all such manual transformations suffer from these problems.
%(In fact, the log-logistic transformation described above might be considered an exceptionally simple case.)
%There is a large body of literature on the theory of automating these transformations,
%but relatively little on improving the programmer's experience of dealing with floating point numbers.
%This pearl focuses on the latter.
This pearl presents a method for alleviating these problems.
Specifically, we will see how to write a function
\begin{lstlisting}
stabilize :: (Floating x, Ord x)
    => (x -> x) -> (x -> x)
\end{lstlisting}
that automatically takes an input function like |logLogistic1| and outputs a numerically stable version of that function.
Unfortunately, this introduces a third problem:

\begin{problem}[unlawfulness]
%How do we know what transformations are valid?
The standard way to encode valid program transformations in Haskell is through type class laws.
A classic example is the |Semigroup| type class
\begin{lstlisting}
class Semigroup a where
    (+) :: a -> a -> a
\end{lstlisting}
which has the law that |(+)| should be associative.
%That is, for every type |a| that is a valid instance of |Semigroup|,
%|(a1+a2)+a3::a| should be equal to |a1+(a2+a3)::a|.
That is, if the type |a| is an instance of |Semigroup|,
and the variables |a1|, |a2|, and |a3| all have type |a|,
then |(a1+a2)+a3| should be equal to |a1+(a2+a3)|.
This is not true of floating point numbers, however.
The expression |(0.1+0.2)+0.3| is famously not equal to |0.1+(0.2+0.3)| for |Double|s.
But there is clearly a sense in which |Double|s are approximately associative,
and performing an associativity transformation cannot change our answer ``too much.''
%How can we formalize this notion?
In order for our |stabilize| method to be generic and useful,
we need a way to formalize this notion.
\end{problem}

This pearl attempts to solve all three problems.
First, we provide a mechanical procedure for constructing FAlgebras from what we call \emph{homogeneous} type classes.
(Both terms will be defined formally in the next section.)
These FAlgebras make our numeric functions \emph{homoiconic};
that is, we can easily get a syntax tree representing the function.
Second, we show how these syntax trees let us easily manipulate our numeric functions to implement the |stabilize| function described above.
Third, we introduce a notion of \emph{topological class laws}.
These class laws let us formalize the intuition that floating point numbers approximately obey class laws.
Finally, we extend our construction of FAlgebras to the \emph{heterogeneous} case.
Heterogeneous type classes let us represent the complicated numeric type class hierarchies that arise in practice when working with numeric types.

\section{Homogeneous Functions are Homoiconic}
Consider the simple numeric class hierarchy shown in Figure \ref{code:group}.
With this class hierarchy, we can write a function that measures the degree to which the |(+)| operation obeys the associative rule.
\begin{lstlisting}
associator :: Group a => a -> a -> a -> a
associator a1 a2 a3
    = ((a1+a2)+a3) - (a1+(a2+a3))
\end{lstlisting}
We call the |associator| function \emph{homogeneous} because:
\begin{enumerate}
\item
The function's return type and all of its parameters are the same type variable.
\item
The constraints contain exactly one type class, and
that type class has exactly one parameter.
\end{enumerate}
Notice that each of the class methods for |Semigroup|, |Monoid|, and |Group| are also homogeneous.
Therefore, we will also say these classes are homogeneous.

In this section, we will see how to convert any homogeneous function into an isomorphic abstract syntax tree (AST).
This ability is called \emph{homoiconicity},
and is usually associated with Lisps.
We will then see several methods for evaluating these ASTs.
In subsequent sections, we will manipulate these ASTs to make floating point operations more convenient.

\begin{figure}
\begin{lstlisting}
class Semigroup a where
    (+) :: a -> a -> a

class Semigroup a => Monoid a where
    zero :: a

class Monoid a => Group a where
    (-) :: a -> a -> a
    negate :: a -> a
\end{lstlisting}
\caption{A simple numeric hierarchy that obeys the restrictions for homogeneous FAlgebras.}
\label{code:group}
\end{figure}

%\begin{figure}
%\begin{lstlisting}
%class Num a where
    %(+), (-), (*)       :: a -> a -> a
    %negate              :: a -> a
    %abs, signum         :: a -> a
    %fromInteger         :: Integer -> a
%
%class Num a => Fractional a where
    %(/)                 :: a -> a -> a
    %recip               :: a -> a
    %fromRational        :: Rational -> a
%
%class Fractional a => Floating a where
    %pi                  :: a
    %exp, log, sqrt      :: a -> a
    %(**), logBase       :: a -> a -> a
    %sin, cos, tan       :: a -> a
    %asin, acos, atan    :: a -> a
    %sinh, cosh, tanh    :: a -> a
    %asinh, acosh, atanh :: a -> a
%
%class Eq a where
    %(==), (/=)          :: a -> a -> Bool
%
%class Eq a => Ord a where
    %compare         :: a -> a -> Ordering
    %(<),(<=),(>),(>=)   :: a -> a -> Bool
    %max, min            :: a -> a -> a
%\end{lstlisting}
%\caption{
    %A portion of the numeric and comparison hierarchies defined in GHC's Prelude.
    %Many class methods satisfy the requirements for homogeneous FAlgebras, except:
    %%|fromInteger| and |fromRational| have non-variable parameters;
    %%and |(==)|, |(/=)|, |(<)|, |(<=)|, |(>)|, and |(>=)| have non-variable return types.
%}
%\label{code:ghc}
%\end{figure}

\subsection{A type class for FAlgebras}
Our construction is closely related to other constructions involving initial algebras, FAlgebras, and free monads.
See for example \cite{swierstra2008}.
The heart of our construction is the |HomFAlgebra| type class, defined as
\begin{lstlisting}
class
    ( Functor (HomSig alg)
    , Foldable (HomSig alg)
    ) => HomFAlgebra (alg::Type->Constraint)
        where
    data HomSig alg a
    runHomSig :: alg a => HomSig alg a -> a
\end{lstlisting}
The |HomFAlgebra| class is unusual in that its parameter has kind |Type->Constraint|.
This means that instances of |HomFAlgebra| will be other type classes (instead of types or type constructors).
The |Hom| in |HomFAlgebra| stands for homogeneous.
Most of the names used in this section will be prefixed with |Hom| to differentiate them from related names used in the later heterogeneous constructions.

The easiest way to explain the |HomFAlgebra| class is by example.
We'll start with the |Semigroup| type class defined above.
\begin{lstlisting}
instance HomFAlgebra Semigroup where
    data HomSig Semigroup a = HomSig_plus a a
    runHomSig (HomSig_plus a1 a2) = a1+a2
\end{lstlisting}
The |HomSig| associated data family is meant to encode what mathematicians call the \emph{signature} of the FAlgebra.
That is, for every class method, |HomSig| should have a constructor with the same number of parameters.
Since |Semigroup| has a single function that takes two parameters,
the |HomSig Semigroup| data instance has a single constructor that takes two parameters.
The |runHomSig| function evaluates the signature.
For each constructor in |HomSig|, |runHomSig| should apply the corresponding function to the constructor's parameters.

To complete the |HomFAlgebra Semigroup| instance, we must also define the following superclass instances.
\begin{lstlisting}
instance Functor (HomSig Semigroup) where
    fmap f (HomSig_plus a1 a2)
        = HomSig_plus (f a1) (f a2)

instance Foldable (HomSig Semigroup) where
    foldr f b (HomSig_plus a1 a2)
        = f a2 (f a1 b)
\end{lstlisting}
As usual, there is only a single valid |Functor| instance.
There are multiple valid |Foldable| instances depending on the order in which |foldr| considers the constructors' parameters,
but for our purposes the order is not important.

Instances get slightly more complicated when the parameter class has superclasses.
In this case, the |HomSig| data instance must also have a constructor for each superclass that stores the superclass's signature.
This can be seen in the |Monoid| and |Group| instances below.
%The |HomFAlgebra| instances of |Monoid| and |Group| demonstrate this.
\begin{lstlisting}
instance HomFAlgebra Monoid where
    data HomSig Monoid a
        = HomSig_zero
        | HomSig_Semigroup_Monoid
            (HomSig Semigroup a)
    runSig HomSig_zero = zero
    runSig (HomSig_Semigroup_Monoid s)
        = runHomSig s

instance HomFAlgebra Group where
    data HomSig Group a
        = HomSig_negate a
        | HomSig_minus a a
        | HomSig_Monoid
            (HomSig_Monoid_Group a)
    runSig (HomSig_negate a) = a
    runSig (HomSig_minus a1 a2) = a1-a2
    runSig (HomSig_Monoid_Group s)
        = runHomSig s
\end{lstlisting}
Notice that the data instance |HomSig Group| does not need a constructor for the |Semigroup| signature because this is included in the |Monoid| signature.
The corresponding |Functor| and |Foldable| instances have been omitted for brevity,
but can be found in the paper's git repo.

%The procedure for generating |HomFAlgebra| instances is purely mechanical.
%The ?? library automates the process with the |mkHomFAlgebra| template Haskell function.

\subsection{Constructing the AST}

We can convert the signature of an FAlgebra into an AST as follows:
\begin{lstlisting}
type HomAST alg a = HomFree (HomSig alg) a

data HomFree f a
    = HomPure a
    | HomFree (f (HomFree f a))
\end{lstlisting}
Intuitively,
the |HomPure| constructor represents a leaf in the AST,
and the |HomFree| constructor represents a branch.
The parameter to |HomFree| is filled by the appropriate |HomSig| data instance.
The constructor used for the |HomSig| instance corresponds to an operation,
and the parameters to the constructor will contain ASTs nested recursively.

For example, consider the expression |(1+2)+3|.
We can create a corresponding AST using |Semigroup|'s |HomFAlgebra| instance.
\begin{lstlisting}
HomFree
  (HomSig_plus
    (HomFree
      (HomSig_plus
        (HomPure 1)
        (HomPure 2)
      )
    )
    (HomPure 3)
  )
\end{lstlisting}
We can also create a corresponding AST for any class that has |Semigroup| as a superclass.
Below is the |Group| AST.
\begin{lstlisting}
HomFree
  (HomSig_Monoid_Group
    (HomSig_Semigroup_Monoid
      (HomSig_plus
        (HomFree
          (HomSig_Monoid_Group
            (HomSig_Semigroup_Monoid
              (HomSig_plus
                (HomPure 1)
                (HomPure 2)
              )
            )
          )
        )
        (HomPure 3)
      )
    )
  )
\end{lstlisting}
Notice that the |HomSig_plus| constructor must be embedded into the |Group|'s |HomSig| with calls to |HomSig_Monoid_Group| and |HomSig_Semigroup_Monoid|.

Constructing these ASTs by hand is a tedious process.
There is a considerable amount of boilerplate in the |Group| AST above as a result of the need to embed the |HomSig_plus| constructor.
We can avoid this boilerplate with the following type class.
\begin{lstlisting}
class
    ( HomFAlgebra alg1
    , HomFAlgebra alg2
    ) => HomView alg1 alg2
        where
    embedHomSig
        :: HomSig alg1 a -> HomSig alg2 a
    unsafeExtractHomSig
        :: HomSig alg2 a -> HomSig alg1 a
\end{lstlisting}
Instances of the |HomView| class should satisfy the property that |alg1| is either equal to |alg2| or is a superclass of |alg2|, and
all such relationships should have a corresponding instance.
For example, |Semigroup| is a superclass of |Group|,
so we should have a |HomView Semigroup Group| instance.
\begin{lstlisting}
instance HomView Semigroup Group
    embedHomSig s
        = HomSig_Monoid_Group (embedHomSig s)
    unsafeExtractHomSig (HomSig_Monoid_Group s)
        = unsafeExtractHomSig s
\end{lstlisting}
Because of this property, we know that the |HomSig alg1 a| data instance can always be embedded into a |HomSig alg2 a| data instance.
But it is only sometimes true that we can extract a |HomSig alg1 a|,
which is why the corresponding function is labeled |unsafe|.

Armed with the |HomView| class, we can create a uniform representation for all class's ASTs.
The expression |(1+2)+3| can now be represented as:
\begin{lstlisting}
HomFree
  (embedHomSig
    (HomSig_plus
      (HomFree
        (embedHomSig
          (HomSig_plus
            (HomPure 1)
            (HomPure 2)
          )
        )
      )
      (HomPure 3)
    )
  )
\end{lstlisting}

%Every FAlgebra can be embedded into itself.
%\begin{lstlisting}
%instance HomView alg alg where
    %embedSig = id
    %unsafeExtractSig = id
%\end{lstlisting}

%\begin{lstlisting}
%instance View Semigroup Monoid where
    %embedSig = Sig_Monoid_Semigroup
    %unsafeExtractSig (Sig_Monoid_Semigroup s) = s

%instance View Monoid Group where
    %embedSig s = Sig_Group_Monoid (embedSig s)
    %unsafeExtractSig (Sig_Group_Monoid s) = unsafeExtractSig s
%\end{lstlisting}

%\begin{lstlisting}
%instance View Semigroup Group where
    %embedSig s = Sig_Group_Monoid (embedSig s)
    %unsafeExtractSig (Sig_Group_Monoid s) = unsafeExtractSig s
%\end{lstlisting}

Using our uniform representation of an AST,
we can make the AST itself an instance of any homogeneous type class.
Again, we'll approach the procedure by example.
We can write a |Semigroup| instance for |HomAST Semigroup a| as:
\begin{lstlisting}
instance
    ( HomView Semigroup alg
    ) => Semigroup (HomFree (HomSig alg) a)
        where
    (+) e1 e2 = HomFree $ embedHomSig $
            HomSig_plus e1 e2
\end{lstlisting}
For each function in the type class,
we simply embed the function's constructor into the AST by prepending the expression |HomFree $ embedHomSig $|.
The |Monoid| and |Group| instances follow a similar pattern.
\begin{lstlisting}
instance
    ( HomView Semigroup alg
    , HomView Monoid alg
    ) => Monoid (HomFree (HomSig alg) a)
        where
    zero = HomFree $ embedHomSig $
        HomSig_zero

instance
    ( HomView Semigroup alg
    , HomView Monoid alg
    , HomView Group alg
    ) => Group (HomFree (HomSig alg) a)
        where
    negate e  = HomFree $ embedHomSig $
        HomSig_negate e
    (-) e1 e2 = HomFree $ embedHomSig $
        HomSig_minus e1 e2
\end{lstlisting}
We can easily create ASTs using these instances.
The |Semigroup| AST can be created with the expression
\begin{lstlisting}
(HomPure 1+HomPure 2)+HomPure 3
    :: HomAST Semigroup a
\end{lstlisting}
And the |Group| AST can be created by just changing the type signature.
\begin{lstlisting}
(HomPure 1+HomPure 2)+HomPure 3
    :: HomAST Group a
\end{lstlisting}

\subsection{Showing the AST}
Now that we can create our ASTs, we want to actually do something with them!
The simplest task is to convert them into a |String| for display.
This is an easy procedure of writing |Show| instances for the |HomSig| and |HomFree| data types.

The are no tricks in the |Show| instance for |Semigroup|'s |HomSig|.
\begin{lstlisting}
instance Show a =>
    Show (HomSig Semigroup a)
        where
    show (HomSig_plus a1 a2)
        = show a1++"+"++show a2
\end{lstlisting}
For each operation constructor, we simply show the parameters and the operation connecting them.
For type classes with superclasses, we also need to recursively show the superclasses as can be seen in the |Monoid| and |Group| instances below.
\begin{lstlisting}
instance Show a =>
    Show (HomSig Monoid a)
        where
    show (HomSig_Semigroup_Monoid s)
        = show s
    show (HomSig_zero) = "zero"

instance Show a =>
    Show (HomSig Group a)
        where
    show (HomSig_Monoid_Group s)
        = show s
    show (HomSig_negate a)
        = "negate "++show a
    show (HomSig_minus a1 a2)
        = show a1++"-"++show a2
\end{lstlisting}
The |HomFree| type's |Show| instance is similarly straightforward..
\begin{lstlisting}
instance (Show a, Show (f (HomFree f a)))
    => Show (HomFree f a)
        where
    show (HomPure a) = show a
    show (HomFree f) = "("++show f++")"
\end{lstlisting}
If the syntax tree contains only a single leaf, we just show that leaf.
If the syntax tree contains a branch,
then put parenthesis around the branch so that we don't lose information about the order of operations, and show the functor in the middle.

\subsection{Evaluating the syntax tree}
Another useful task for ASTs is to evaluate them,
generating the value they represent.
This data type is called the free monad because as long as the parameter |f| is a |Functor|,
then |HomFree| has a valid |Monad| instance.
Since all |Monad|s are by necessity also |Functor|s,
|HomFree| has a valid |Functor| instance as well.
\begin{lstlisting}
instance Functor f => Functor (HomFree f) where
    fmap g (HomPure a) = HomPure (g a)
    fmap g (HomFree f)
        = HomFree (fmap (fmap g) f)
\end{lstlisting}
For our purposes, that's all the structure we'll need.

First, we construct a function that ``runs'' the syntax tree, converting it into a single value.
\begin{lstlisting}
runHomAST :: (HomFAlgebra alg, alg a)
    => HomAST alg a -> a
runHomAST (HomPure a) = a
runHomAST (HomFree f) = runHomSig (fmap evalHom f)
\end{lstlisting}
If our syntax tree already consists of just a single value (i.e. it is the |HomePure| constructor),
then we just return that value.
Otherwise, we use the |Functor| instance to recursively convert the |HomSig|s into types of |a|, then a call to |runHomSig| combines the results.

\subsection{Homoiconicity at last!}
We are finally ready to see that our construction makes homogeneous functions homoiconic.
If we open GHCi and enter the following expression
\begin{lstlisting}
ghci> let ast=associator (HomPure a1) (HomPure a2) (HomPure a3)
\end{lstlisting}
the display shows our AST
\begin{lstlisting}
\end{lstlisting}
We can also evaluate our AST and see that doing so gives the same results as running the original expression.
\begin{lstlisting}
ghci> runHomAST ast

ghci> associator a1 a2 a3

\end{lstlisting}
We are now ready to use our ASTs to make working with numeric functions easier.
%\subsection{The Prelude's Numeric Hierarchy}

\section{Implementing the stabilize function}

We've seen how to easily construct the ASTs,
now it's time to perform computations on them.
The standard way to deconstruct an AST is via pattern matching.
We'll use the |ViewPatterns| and |PatternSynonyms| language extensions to simplify this task.
Then we'll see three examples of improving floating point accuracy.

\subsection{Pattern matching the AST}

We'll begin with a very simple example.
Consider the following simple, but numerically unstable function.
\begin{lstlisting}
testFunc1 :: Floating a => a -> a
testFunc1 = log(exp a)
\end{lstlisting}
As in our |logLogistic1| function, |testFunc| is unstable when specialized to |Double|s
because input values greater than (approximately) 710 cause the |exp| function to overflow to infinity;
the log of infinity is still infinity;
so the returned answer is infinity.
But clearly this is not the correct answer.
Mathematically, the $\log$ is the inverse of $\exp$, so the final result should be 711.

First, we'll see how to properly manipulate the AST to fix this problem.
%We need a function of type |AST Floating a -> AST Floating a|.
We're trying to convert the expression |log (exp x)| into simply |x|.
To do this, we can just pattern match on the AST for the left hand side of the expression,
and replace it with the right hand side.
\begin{lstlisting}
logexpAST1 ::
    AST Floating a -> AST Floating a
logexpAST1
    (Free (Sig_log (Free (Sig_exp a)))) = a
logexpAST1 a = a
\end{lstlisting}
This function has two problems.
First, our pattern matching expression is large and obnoxious.
Second, it's not generic.

We can fix the second problem using the |ViewPatterns| language extension.
\begin{lstlisting}
logexpAST2 :: View Floating alg
    => AST alg a -> AST alg a
logexpAST2
    (Free (unsafeExtractSig -> Sig_log
    (Free (unsafeExtractSig -> Sig_exp a))))
    = a
logexpAST2 a = a
\end{lstlisting}
But this made our pattern matches even uglier!
Fortunately, we can use the |PatternSynonyms| extension to greatly simplify these patterns.
\footnote{
    %These patterns can be made bidirectional using the |embedSig| function.
    For ease of presentation, only the unidirectional case is shown.
}
\begin{lstlisting}
pattern AST_log ::
    ( View Floating alg
    ) => AST alg a -> AST alg a
pattern AST_log e
    <- Free (unsafeExtractSig -> Sig_log e)

pattern AST_exp ::
    ( View Floating alg
    ) => AST alg a -> AST alg a
pattern AST_exp e
    <- Free (unsafeExtractSig -> Sig_exp e)
\end{lstlisting}
Armed with these pattern synonyms, we can rewrite our function in the much simpler form below.
\begin{lstlisting}
logexpAST3 :: View Floating alg
    => AST alg a -> AST alg a
logexpAST3 (AST_log (AST_exp a)) = a
logexpAST3 a = a
\end{lstlisting}
Now that we can manipulate the AST, let's apply it to our function.
\begin{lstlisting}
transform1 :: forall alg x.
    ( FAlgebra alg
    , alg x
    ) => (AST alg x -> AST alg x)
      -> (AST alg x -> AST alg x)
      -> (x -> x)
transform1 go f x = runHomAST $ go $ f (Pure x)
\end{lstlisting}


But is our function really correct?
\begin{lstlisting}
testFunc2 :: Floating a => a -> a
testFunc2 = 1+log(exp a)
\end{lstlisting}
What if we call

\begin{lstlisting}
logexpAST4 :: View Floating alg => AST alg a -> AST alg a
logexpAST4 (AST_log (AST_exp a)) = a
logexpAST4 (Free f) = Free (fmap logexpAST4 f)
logexpAST4 (Pure a) = Pure a
\end{lstlisting}

\begin{lstlisting}
testFunc :: Floating a => a -> a
testFunc a = log(exp a)

ghci> logexpAST4 (testFunc 2) :: AST Floating Double
((fromInteger 1)+(fromInteger 2))

testFunc2 :: Floating a => a -> a
testFunc2 a = 1+exp(exp(exp(log(log(log a)))))

ghci> fixAST stabilizeAST (testFunc2 var1 :: AST Floating Var)
((fromInteger 1)+var1)
\end{lstlisting}

\begin{lstlisting}
fixAST :: (Eq (Sig alg (Free (Sig alg) a)), Eq a) => (AST alg a -> AST alg a) -> AST alg a -> AST alg a
fixAST f ast = if ast==ast'
    then ast
    else fixAST f ast'
    where
        ast' = f ast
\end{lstlisting}

\subsection{More complex rules}

The rules in in the previous section are relatively simple.
They could just as easily have been written using GHC's rewrite rules syntax.
Rewrite rules are the primary mechanism for optimising source code in GHC.
One optimization that rewrite rules cannot support is constant folding,
and there is code in GHC specially to perform this constant folding.

\begin{lstlisting}
foldConstants :: View Num alg => AST alg a -> AST alg a
foldConstants (AST_plus   (AST_fromInteger a1) (AST_fromInteger a2)) = AST_fromInteger (a1+a2)
foldConstants (AST_minus  (AST_fromInteger a1) (AST_fromInteger a2)) = AST_fromInteger (a1-a2)
foldConstants (AST_mul    (AST_fromInteger a1) (AST_fromInteger a2)) = AST_fromInteger (a1*a2)
foldConstants (AST_negate (AST_fromInteger a1)) = AST_fromInteger (negate a1)
foldConstants (Free sig) = Free (fmap foldConstants sig)
foldConstants (Pure a  ) = Pure a
\end{lstlisting}

\begin{lstlisting}
constExpr :: Num a => a
constExpr = 4+2*(8-2)-1

-- ghci> fixAST foldConstants constExpr
-- (fromInteger 18)

constFunc :: Num a => a -> a -> a
constFunc x1 x2 = x1*2-(7-2)*x2

-- ghci> fixAST foldConstants (constFunc 2 3)
-- (fromInteger -11)
-- ghci> fixAST foldConstants (constFunc var1 var2)
-- ((var1*(fromInteger 2))-((fromInteger 5)*var2))
\end{lstlisting}

\subsection{The log-logistic function}

\begin{lstlisting}
stabAST :: Eq a => AST FloatingOrd a -> AST FloatingOrd a
stabAST
    (AST_log
        (AST_div
            (AST_fromInteger 1)
            (AST_plus
                (AST_fromInteger 1)
                (AST_exp
                    (AST_negate x)
                )
            )
        )
    )
    = logLogistic2 x
\end{lstlisting}

\section{Making floats lawful}

\subsection{Varieties and exact laws}

A class law can be defined by a pair of ASTs that substituted for each other without changing the semantics of a program.
For convenience, we assign each law a name and put it in the following data type.
\begin{lstlisting}
data Law alg = Law
    { lawName :: String
    , lhs     :: HomAST alg Var
    , rhs     :: HomAST alg Var
    }
\end{lstlisting}
A \emph{variety} is an algebraic structure that has a set of associated laws.
We can encode a variety in Haskell as:
\begin{lstlisting}
class HomFAlgebra alg => HomVariety alg where
    laws :: [Law alg]
\end{lstlisting}
Again, notice that instances of this class will be other type classes.

To encode the laws of a type class, we create an appropriate |HomVariety| instance.
The |Semigroup| class has the following instance defining what it means to be associative.
\begin{lstlisting}
instance HomVariety Semigroup where
    laws = [ Law
        { lawName = "associative"
        , lhs = (var1+var2)+var3
        , rhs = var1+(var2+var3)
        } ]
\end{lstlisting}

\subsection{Listing the laws}

When working with a large numerical hierarchy,
it is easy to forget the exact laws that a class must obey.
Wouldn't it be nice if there were an easy way to ask the repl what the class laws are?

With this encoding, there is.
First, we define a function that pretty prints a |Law|.
\begin{lstlisting}
printLaw :: Show (HomAST alg Var)
    => Law alg -> IO ()
printLaw law = do
    putStrLn ( "  "++lawName law++":"      )
    putStrLn ( "    lhs: "++show (lhs law) )
    putStrLn ( "    rhs: "++show (rhs law) )
\end{lstlisting}
Then we define a function that calls |printLaw| on each class law.
\begin{lstlisting}
printLaws :: forall alg.
    ( HomVariety alg
    , Show (HomAST alg Var)
    ) => Proxy alg -> IO ()
printLaws palg = do
    forM_ (laws::[Law alg]) printLaw
\end{lstlisting}
Now, if we enter the command |printLaws @Semigroup| into the REPL,
we get a nice reminder that every instance of |Semigroup| should be associative.

\subsection{The floating point topology}

Checking if two floating point numbers are equal is almost always a bad idea.
Better practice is to check if the two numbers are ``close enough'' for your application.
We can formalize this notion via topology.

In mathematics, a \emph{topology} is a set $X$ equipped with a family of subsets of $X$ called $\tau$ such that:
%In mathematics, a \emph{topology} is a set $X$ and a set of neighborhoods $\tau \subseteq 2^X$ such that:
\begin{enumerate}
\item
$\tau$ contains the empty set and the set $X$.
\item
$\tau$ is closed under an arbitrary number of unions.
\item
$\tau$ is closed under finitely many intersections.
\end{enumerate}
We call an element of $\tau$ a neighborhood.

To encode the notion of a topology in Haskell,
we first must encode the idea of a set.

\begin{lstlisting}
class Poset a where
    inf :: a -> a -> a
\end{lstlisting}

\begin{lstlisting}
class Poset (Neighborhood a) => Topology a
        where
    type Neighborhood a
    (==) :: a -> a -> (Neighborhood a -> Bool)
\end{lstlisting}

%It is well known that equality testing is almost always a bad idea for floating point numbers.
As mentioned before, floating point numbers are not strictly associative,
but they have a notion of being approximately associative.
In particular, when we call the |associator| function defined above on any three floating point values,
we can guarantee that the result is small relative to the input values.

\subsection{Topological test cases}

\section{Conclusion}

There are two more points worth discussion.

The first is efficiency.
The |stabilize| function is absurdly slow.
If you view the core, you can see two very different outputs.
But the output of |stabilize| is highly structured and ready to be optimized.
It's left as an exercise for the reader to write a GHC plugin to fix the problem.

The second is improved class hierarchies.
Most of the interesting alternative numeric hierarchies involve many non-homogeneous functions.
In particular, they use type classes.


%%%%%%%%%%%%%%%%%%%%%%%%%%%%%%%%%%%%%%%%%%%%%%%%%%%%%%%%%%%%%%%%%%%%%%%%%%%%%%%%

%\appendix
%\section{Appendix Title}
%
%This is the text of the appendix, if you need one.
%Most notably, it is not generic.
%
%\acks
%
%Acknowledgments, if needed.
%

%%%%%%%%%%%%%%%%%%%%%%%%%%%%%%%%%%%%%%%%%%%%%%%%%%%%%%%%%%%%%%%%%%%%%%%%%%%%%%%%

\bibliographystyle{abbrvnat}
\bibliography{paper}


\end{document}
